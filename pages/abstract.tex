% !TeX root = ../main.tex
\chapter{\abstractname}

The B-Tree data structure is a classical data structure
that builds the foundation of many modern database systems.
Since its first definition in \parencite{DBLP:journals/acta/BayerM72}
variations have been developed.
More complex operations such as deletion are famously spared by many
standard references.
The original definition and modern variants are examined and compared
with mechanically verified versions of the data structure.
A functional version that is promising to analyse in the Isabelle/HOL framework
is then implemented followed by a proof of its correctness with respect to refining
a set on linearly ordered elements.
A proof of the logarithmic relationship between height and number of nodes
based on the original paper was given to point at the efficiency of
operations on the tree.

From the functional specification an imperative specification
derived that stores and retrieves nodes from an abstracted heap.
The specification is shown to again refine the functional program
using the separation logic utilities for the Isabelle Refinement Framework from
\parencite{DBLP:journals/jar/Lammich19}.
This way an imperative implementation of the B-Tree data structure is obtained
that is proven to correctly handle insertions and element checks.


%TODO: Abstract


