% !TeX root = ../main.tex
% Add the above to each chapter to make compiling the PDF easier in some editors.

\chapter{Imperative B-Trees in Isabelle}\label{chapter:imp-set}

In the previous chapter, we have seen
an abstract definition of B-Trees and the reasoning
behind its correct implementation of the set interface.
However, the specification would not yield 
efficiently executable code.
% main difference: specify allocation of data
The reason is that the abstract specification works with
the persistent datatypes of nodes and lists.
This way, trivially no data is unknowingly
modified or corrupted,
however this is at the cost of computational efficiency
as changes require do allocate memory for completely new object.

In order to obtain an efficient implementation
on ephemeral data structures common to imperative languages,
we specify imperative code and show that
it refines the abstract specification.
The imperative code can then be translated to
the languages Scala \parencite{OderskyScala} and SML \parencite{DBLP:books/daglib/0069232}
using the code generation
facilities in Isabelle/HOL \parencite{Haftmann20codegeneration}.
However, since these facilities are extensible,
code extraction is not limited
to these languages.

\section{Refinement to Imperative/HOL}

The proofs of correct behavior for our imperative
code are stated separation logic \parencite{DBLP:conf/lics/Reynolds02},
an extension of Hoare Logic.
It was formalized in Isabelle/HOL by \parencite{DBLP:journals/jar/Lammich19}
and comes with a framework that simplifies reasoning about
its expressions.

\subsection{Separation Logic}

Separation logic provides a way to reason about mutable resources
that lie in separated parts of an external heap,
an abstracted memory device.
The assumptions on the state of the heap are called \textit{assertions}.
They are specified as formulae that hold for specific heap states.
The basic assertions used in this work follow.

\begin{itemize}
    \item \textit{emp} holds for the empty heap
    \item \textit{true} and \textit{false} hold for every and no heap respectively
    \item $\uparrow(P)$ holds if the heap is empty and predicate $P$ holds
    \item $a \mapsto_a as$ holds if the heap at position $a$ is reserved and contains
    an array representing $as$.
    \item $a \mapsto_r x$ holds if the heap at position $a$ is reserved and contains
    value $x$ where $x$ is of some type $'a::heap$
    \item $\exists_A x.\ P\ x$ holds if there exists some $x$ such that predicate P
    holds on the heap for given $x$.
    \item $P_1 * P_2$ holds if each assertion holds on its part of the heap
    and the areas of the heap described by each assertion are non-overlapping
\end{itemize}

With the tools provided by separation logic,
we well prove our imperative implementations correct
by showing that they refine the abstract implementation in \Cref{chapter:abs-set}.

In general, refinement relates a concrete data structure
to an abstract structure via some refinement assertion.
Examples for such a refinement assertion are the id assertion and the
list assertion.
The latter relates lists, given a refinement
assertion between the elements of the first and second list.

\begin{lstlisting}[mathescape=true, language=Isabelle]
definition id_assn a b = $\uparrow$(a = b)

fun list_assn :: ('a $\Rightarrow$ 'c $\Rightarrow$ assn) $\Rightarrow$ 'a list $\Rightarrow$ 'c list $\Rightarrow$ assn where
  list_assn P [] [] = emp
| list_assn P (x#xs) (y#ys) = P x y * list_assn P xs ys
| list_assn _ _ _ = false
\end{lstlisting}

We can then specify Hoare Triples on assertions,
to encapsulate statements about imperative programs.

\begin{definition}{Hoare Triple}
    $\langle$ P $\rangle$ c $\langle \lambda r. $ Q r $\rangle$ \\
    holds iff if assertion $P$ holds on some heap before operation $c$
    is executed on the heap,
    and operation $c$ returns some $r$, then assertion $Q$ holds
    for $r$ and on the modified heap
\end{definition}

A simple example Hoare Triple is 
$\langle$ id\_assn $a$ $b$ $\rangle$ return $b$ $\langle \lambda r.$ id\_assn $a$ $r$ $\rangle$.
In the refinement process, we will use more complex relationship assertions
on the input $b$ to some object $bi$ and show that the value $r$ returned
by the imperative program satisfies a relationship to the output of the refined
function applied to $bi$.
Further we use the shorthand notation $\langle\ P\ \rangle\ c\ \langle \lambda r.\ Q\ r\ \rangle_t$
which stands for $\langle\ P\ \rangle\ c\ \langle \lambda r.\ Q\ r\ * true\ \rangle$.
Since $true$ holds for any heap, this part of the statement may be used to subsume
all temporary and discarded variables.

Refining all set operations from \Cref{chapter:abs-set}
will yield imperative code that operates on the heap
with option to use efficient destructive updates.
We do not need to show that the operations themselves
satisfy the requirements of the set interface.
Instead we merely need to show that the imperative code
acts on a concretized version of the B-Trees
the same way as the abstract operations on the abstract version.

% TODO note what to introduce here
% abstract idea: implementation that has same effect on concrete ds
% as abstract function on abstract ds
% TODO short intro hoare triples and notation here
% *  and \uparrow

\subsection{Refinement of abstract lists}

In the imperative collection framework\parencite{DBLP:journals/jar/Lammich19},
the abstract datatype list is usually
refined to a dynamic array like data structure.
It comprises an array and a natural number,
where the latter denotes the current number of elements
stored in the array.
B-Tree nodes do not always contain a constant number of elements,
which is why this property should be transferred to
the data structure refining key-value lists. 

% TODO intended physical storing of B-Tree on pages of memory
% --> partially filled array, non-dynamic!
However, dynamic arrays are designed to grow and shrink
as elements are inserted and deleted.
B-Trees were invented to be stored on slow hard disks.
As one disk access loads a chunk of data into a page of main memory,
nodes of roughly this size make the most efficient use of the loading times.
\parencite{DBLP:journals/acta/BayerM72}
%TODO give clear reason
Therefore the data structure to store the node content
should reserve a fixed amount of data, independent
of the actual number of keys and children.

Based on the definition of dynamic arrays in the 
imperative collection framework,
we therefore introduce a simpler data structure,
the \textit{partially filled array},
that keeps count of currently inserted elements,
but does neither grow nor shrink.
Abstractly, we refine lists in the functional B-Trees to this
data-structure.
A list $l$ is represented by a partially filled array $(a,n)$,
if the array $a$ represents a list $l'$, of which the first $n$
elements form list $l$.
Formally,

\begin{lstlisting}[mathescape=true, language=Isabelle]

datatype 'a pfarray = 'a array $\times$ nat

definition is_pfa where c l $\equiv$
    $\lambda$(a,n). $\exists_A$ l'. a $\mapsto_a$ l' *  $\uparrow$(c = length l' $\wedge$ n $\le$ c $\wedge$ l = (take n l'))
\end{lstlisting}

where $c$ is the $capacity$ of the array, that is the actual allocated size
of the array on the heap.
All references to lists in the abstract datatype $btree$
will in the imperative version be refined to partially filled arrays.

However one should note that
from the set operation specification in \Cref{chapter:abs-set} alone
it is not determined which operations are supposed to be conducted
in-place and which will require copying data.
To obtain the most efficient code we therefore manually
define the following operations to replace complex composite
operations based on list construction and concatenation.

% abstract function has length, insertion, take/drop
% --> concrete implementations
\begin{figure}
\begin{lstlisting}[mathescape=true, language=Isabelle,label={lst:pfarray-def},
    caption={Important Partially Filled Array functions for Insertion.}]
definition pfa_length $\equiv$ $\lambda$(a,n). return n

definition pfa_get $\equiv$ $\lambda$(a,n) i. Array.nth a i

definition pfa_set $\equiv$ $\lambda$(a,n) i x. do {
    a $\leftarrow$ Array.upd i x a;
    return (a,n)
}

definition pfa_shrink k $\equiv$ $\lambda$(a,n). return (a,k)

definition pfa_shrink_cap  k $\equiv$ $\lambda$(a,n). do {
    a' $\leftarrow$ array_shrink a k; (* returns an array with the given actual size, potentially reallocated*)
    return (a',min k n)
}

definition pfa_drop $\equiv$ $\lambda$(src,sn) si (dst,dn). do {
    blit src si dst 0 (sn-si);
    return (dst,(sn-si))
}

definition pfa_insert $\equiv$ $\lambda$(a,n) i x. do {
  a' $\leftarrow$ array_shr a i 1; (* shifts elements to the right by 1 *)
  a'' $\leftarrow$ Array.upd i x a;
  return (a'',n+1)
}

definition pfa_ensure $\equiv$ $\lambda$(a,n) k. do {
  a' $\leftarrow$ array_ensure a k default; (* returns an array with given minimal size, potentially reallocated *)
  return (a',n)
}

definition pfa_insert_grow  $\equiv$ $\lambda$(a,n) i x. do {
  a' $\leftarrow$ pfa_ensure (a,n) (n+1);
  a'' $\leftarrow$ pfa_insert a' i x;
  return a''
}

$\dots$
\end{lstlisting}
\end{figure}

% TODO also include what the rule gives us?

During the splitting of nodes, a part of the overflowing
list is copied into another array.
The operation to copy data is based on the function $blit$,
that copies a slice of an array to the starting position of another array.
The function already existed in the standard collection and could be reused
to define the \textit{pfa\_drop} function in \Cref{lst:pfarray-def}, which in this implementation copies data to a new array.

The $blit$ function is however not capable of correctly copying data
within the array, as required for efficient in-place insertion and deletion.
A suitable function, called $sblit$, was hence added in the course of this project.
The relevant addition is to differentiate between copying elements
from higher to lower indices or from lower to higher indices.
Depending on the direction of the copy, it is relevant
in which order to copy singleton elements so that in case
of overlapping ranges no elements are overwritten before being copied.

By adding a function that copies elements in reverse order
and conditionally calling either copy function,
loss-free in-place copying was achieved.
The resulting function is conveniently agnostic to the direction
of the copying, just as the respective implementations
in target languages.\footnote{
    See for example \href{https://caml.inria.fr/pub/docs/manual-ocaml/libref/Array.html}{the specification of \texttt{Array.blit} in Ocaml} or
    the \href{https://smlfamily.github.io/Basis/array-slice.html\#SIG:ARRAY_SLICE.copy:VAL}{\texttt{ArraySlice.copy} specification in SML}.
}
The \textit{pfa\_insert} function in \Cref{lst:pfarray-def} makes
implicit use of in-place copying by calling a function to shift elements
to the right by one.
The element to be inserted is then placed in the remaining free spot.

For all functions in \Cref{lst:pfarray-def},
appropriate Hoare Triples were derived.
All triples are straightforward to derive which is why they are not
listed listed in great detail.
To roughly sketch the obtained lemmas, the two examples
\textit{pfa\_drop} and \textit{pfa\_shrink} are investigated below.
The most important difference is whether
the operations are destructive (in-place) or not.
For most operations, the approach that required the least
memory allocations was chosen.\footnote{
    The only exception is node$_i$, which in the case of overflow
    currently requires temporary allocation of an array that holds more than $2k$ elements.
}

\begin{lemma}
    $k \le \length s \wedge (\length s - k) \le dn \Longrightarrow$ \\
    \begin{center}
    $\langle \pfarraycap$ sn s si * $\pfarraycap$ dn d di $\rangle$ \\
    $\pfadrop$ si k di \\
    $\langle\lambda di'. \pfarraycap$ sn s si * $\pfarraycap$ dn ($\drop$ k s) di' $\rangle$
    \end{center}
\end{lemma}

The array $si$ stays untouched.
This fact is expressed by the fact that the term about its relationship to $s$
is still available in the postcondition of the triple.
The only thing that changed is the content of $di$.\footnote{
    In the actual proof, where possible, we even show that the array itself has not changed
    (i.e. has not been reallocated).
    The correct notation does however not yield high readability
    and is hence abstracted here.
}
The $take$ function was refined differently.
The cheapest and in our context consistent option is
reducing the number of elements that are considered part of the represented list
from $n$ to $k$.
The result is that the original array gets modified and we loose any information
about the content that is written beyond given $k$.

\begin{lemma}
    $k \le \length s\Longrightarrow$ \\
    \begin{center}
    $\langle \pfarraycap$ sn s si$\rangle$
    $\pfashrink$ si k
    $\langle\lambda si'. \pfarraycap$ sn ($\take$ k s) si' $\rangle$
    \end{center}
\end{lemma}

With these operations and suitable Hoare-Triples about them,
we have all the tools required to refine the abstract set implementation.
The only missing piece is an imperative split function.
% TODO more about sblit?

\section{Set-Operations}

With refinements for all basic components in the B-Tree set operations,
the B-Tree datatype itself can be refined.
Rather than actually storing nodes within lists,
the B-Tree nodes are supposed to be stored in the heap,
only storing pointers to other nodes in the element list.
Since the Imperative Refinement Framework does not support null pointers,
B-Tree nodes are refined by heap reference \textit{options} instead.
None represents leafs or empty trees.
The physical B-Tree datatype does hence only need to cover the case
of actually being an inner node.

\begin{lstlisting}[mathescape=true, language=Isabelle]
datatype 'a btnode =
    Btnode ('a btnode ref option*'a) pfarray 'a btnode ref option
\end{lstlisting}

Note the similarity to \Cref{lst:btree-def},
only that \textit{btree} was replaced by \textit{btnode ref option}
and \textit{list} was replaced by \textit{pfarray}.

An abstract B-Tree is represented by such a physical node
if all references on subtrees represent the corresponding
abstract subtrees and all keys are the same.

\begin{lstlisting}[mathescape=true, language=Isabelle]
fun btree_assn :: nat $\Rightarrow$ 'a::heap btree $\Rightarrow$ 'a btnode ref option $\Rightarrow$ assn where
  btree_assn k Leaf None = emp |
  btree_assn k (Node ts t) (Some a) = 
 ($\exists_A$ tsi ti tsi'.
      a $\mapsto_r$ Btnode tsi ti
    * btree_assn k t ti
    * is_pfa (2*k) tsi' tsi
    * list_assn ((btree_assn k) $\times_a$ id_assn) ts tsi'
    ) |
  btree_assn _ _ _ = false

abbreviation blist_assn k $\equiv$ list_assn ((btree_assn k) $\times_a$ id_assn)

\end{lstlisting}

It seems a little bit awkward that the definition
includes two relationships regarding only the list of elements in the subtree.
The reason is that there are two steps of refinements going on.
First all elements in a list are refined to their physical counterpart.
This fact is expressed by the list assumption term in the assumption.
Second, the list itself is refined to an array.
This is expressed in the term about the partially filled array relationship.
Since the list assumption makes recursive use of the
B-Tree assumption, this is however a very convenient
way of expressing the desired relationship.

Using the parameters of \textit{is\_pfa},
note that we can specify that every node of a B-Tree should have
capacity for $2k$ elements.

\subsection{The split function}
\label{sec:imp-split}

To implement any set operations we again need a split operation.
However, what split function exactly will be used can be abstracted again.
The only important thing is that it refines an abstract split function.
Imperative split functions should be efficient and as such not actually
split an array in half and return it the way the abstract function did.
Rather than that the function should return the index of the subtree
in which should be recursed.
The refinement condition derived from this relationship is as follows.

\begin{lstlisting}[mathescape=true, language=Isabelle]
definition split_relation xs $\equiv$ $\lambda$(ls,rs) i. i $\le$ length xs $\wedge$ ls = take i xs $\wedge$ rs = drop i xs
\end{lstlisting}

A very useful alternative statement of the relationship follows automatically.

\begin{lemma}
    $\splitrelation xs\ (ls,rs)\ i = (xs = ls@rs \wedge i = \length ls)$
\end{lemma}

From this, we characterize the desired imperative
split function \texttt{imp\_split} by its Hoare-Triple.

\begin{align*}
    \langle & \pfarraycap c\ tsi\ a * \blistassn k\ ts\ tsi *  true \rangle \\
            & \impsplit a \\
\langle\lambda i. & \pfarraycap c\ tsi\ a * \blistassn k\ ts\ tsi * \uparrow(\splitrelation ts\ (\splitfun ts p)\ i)\rangle_t
\end{align*}

Just as with the abstract implementation,
the imperative refinements can be built with this characterization alone.
However this time we are actually interested in finding an
efficient split and will spend some time finding
a good split function.
%TODO move to after the set operations?

In the imperative context, we prefer not to directly
implement the recursive function from \Cref{fig:linear_split}
but we embark on the exercise of using a while loop.
The implementation in \Cref{lst:imp-linear-split}
is very basic but good for getting to know
the usage of while loops in imperative HOL.
Every iteration, an index on the array is moved forward by one.
The main part is done in the loop head - if we
are within the array bounds,
we obtain the current separator and check if it is
smaller than the partitioning element.
We return the first index, at which the separator is greater or equal.

\begin{figure}
\begin{lstlisting}[mathescape=true, language=Isabelle, caption={The imperative linear split},
    label={lst:imp-linear-split}]
definition lin_split :: ('a::heap $\times$ 'b::{heap,linorder}) pfarray $\Rightarrow$ 'b $\Rightarrow$ nat Heap where
    lin_split $\equiv$ $\lambda$ (a,n) p. do { 
   
        i $\leftarrow$ while  
            ($\lambda$i. if i<n then do { 
            (_,s) $\leftarrow$ Array.nth a i; 
            return (s<p) 
            } else return False)  
            ($\lambda$i. return (i+1))  
            0; 
                
        return i 
    }
\end{lstlisting}
\end{figure}

This function behaves quite different to the abstract linear split
that was implemented before.
Therefore, proving that it refines the abstract function requires a small detour.
We first express and show the most basic, non-trivial result of the function.
With the help of that we may then show that the result is the same as the result
of applying the abstract function.

We begin with stating a simple matching Hoare Triple
for the function.
It follows by supplying the following loop invariant:

\begin{itemize}
    \item $\forall j<i. \snd (xs!j) < p$: All elements up to the current $i$
        are smaller than the partitioning element.
    \item $\uparrow(i \leq n)$: $i$ does not exceed the length of the array and
    \item $\pfarraycap c\ xs\ (a,n)$: the array stays untouched by the operation
\end{itemize}

The measure for the loop is the difference $n-i$ which
decrements by one in each iteration and stays positive.
From this we obtain the Hoare Triple of the whole linear split.

\begin{lemma}
    \begin{align*}
        \langle &\pfarraycap c\ xs\ (a,n)\rangle \\
                  &\linsplit\ (a,n)\ p \\
        \langle \lambda i. & \pfarraycap c\ xs\ (a,n) \\
        * &\uparrow(i \leq n 
            \wedge (\forall j < i. snd (xs!j) < p) 
            \wedge (i<n \longrightarrow snd (xs!i) \geq p)) \rangle_t 
    \end{align*}
\end{lemma}

For unexciting reasons, this post condition
actually suffices to imply that there exists a split relation
to the result of an abstract split function.
Specifically, we use the abstract split function from
\Cref{fig:linear_split}. %TODO or explain why a different function was used
For the two cases of loop termination,
either $i = n$, or $i < n$, we find that
the abstract function will return the exact same split.

This is great to know since the previous lemma
follows directly from the computation and is closely related
to the loop invariant.
\footnote{Another, albeit more theoretical,
reason for excitement is that now we can specify an imperative B-Tree
specification without any reference to the abstraction.
We simply always use the well known split function in the abstraction
and only require a post condition similar to the one just shown
on the imperative split function.}
This will simplify proofs for alternative imperative split functions.
With this assurance we approach the binary split.

The binary split in \Cref{lst:imp-binary-split} conducts a binary search on
the list of elements.
Rather than walking through the array,
it narrows down a windows on the array that becomes smaller with
every iteration.
In the final iteration,
the windows must have narrowed down to a single element,
which is the result.

\begin{figure}
\begin{lstlisting}[mathescape=true, language=Isabelle, caption={The imperative binary split},
    label={lst:imp-binary-split}]
    definition bin_split :: ('a::heap $\times$ 'b::{heap,linorder}) pfarray $\Rightarrow$ 'b $\Rightarrow$ nat Heap 
    where 
      bin_split $\equiv$ $\lambda$(a,n) p. do { 
    (low',high') $\leftarrow$ while  
      ($\lambda$(low,high). return (low < high))  
      ($\lambda$(low,high). let mid = ((low  + high) div 2) in 
       do { 
        (_,s) $\leftarrow$ Array.nth a mid; 
        if p < s then 
           return (low, mid) 
        else if p > s then 
           return (mid+1, high) 
        else return (mid,mid) 
       })  
      (0::nat,n); 
    return low' 
  }
\end{lstlisting}
\end{figure}

A detailed analysis of binary search algorithms
may be found in \parencite{DBLP:journals/csedu/Montague91},
which has served as an orientation for the derived algorithm.
The main idea is that the algorithm iteratively
narrows a window on the array in which the desired element may lie.
In each step, the middle element of that window is inspected
and either desired element is returned or the algorithm
recurses into either left or right remaining subwindow.

Since pairs as list elements
add another layer of complexity for the simplifier,
a binary split algorithm on lists that contain
only single elements has been derived first.
After several small corrections,
it was proven to be correct.
The split function for lists of pairs was then defined
and could be tackled by the same approach,
using the below loop invariant.
Note that the binary split works on two indices
as states, the upper bound $h$ and the lower bound $l$.

\begin{itemize}
    \item $\forall j<l. \snd (xs!j) < p$: All elements up to the current $i$
        are smaller than the partitioning element.
    \item $h<n \rightarrow \snd (xs!h) < p$: If the upper bound is strictly less than the length
    of the array, the element it points to is greater than the partioning element
    \item $l \leq h$: The lower bound is always less or equal to the upper bound
    \item $\uparrow(h \leq n)$: the upper bound does not exceed the length of the array and
    \item $\pfarraycap c\ xs\ (a,n)$: again, the array stays untouched by the operation
\end{itemize}

In this case the window that is considered by the binary split
provides for the loop measure, as $h-l$ is decreased by at least one in each step.
Proving that this was the case was not the main issue,
but more to find a correct formulation of the algorithm
where this is actually true.

Overall obtaining the same Hoare Triple
as for the linear split required only
this slightly more complex loop invariant
and invocation of a number of algebraic lemmas.
This was somewhat of a surprise as the functional
version of a binary split is much harder to
specify and analyse than a linear split.
The additional requirement for the binary split is
naturally that the list of separators must be sorted.

\begin{lemma} $\sorted (\separators xs) \Longrightarrow$ \\
    \begin{align*}
        \langle &\pfarraycap c\ xs\ (a,n)\rangle \\
                  &\binsplit\ (a,n)\ p \\
        \langle \lambda i. & \pfarraycap c\ xs\ (a,n)
        * \uparrow(i \leq n 
            \wedge (\forall j < i. snd (xs!j) < p) 
            \wedge (i<n \longrightarrow snd (xs!i) \geq p)) \rangle_t 
    \end{align*}
\end{lemma}

Since we know that this implies equivalence to the abstract split function
and sortedness of the separators is guaranteed by the sortedness
invariant of the B-Trees,
we can safely use this function for specifying
the set operations.

%TODO Note the choice of a while-loop for efficiency
%TODO The split function can be implemented linearly (easier, proof of concept)
% but due to abstraction of functionality may use binary split as well.

\subsection{The isin function}

In general, the formulation of the imperative set operations
do quite directly follow
from the refinement of the abstract operations.
Pointers need to be dereferenced, modified and updated.
Functions such as length need to be called and have their result
stored explicitly before use.
Other than that, the definition of the imperative isin in \Cref{lst:imp-isin-fun}
function should bare no surprises.
Note that it makes use of the function \textit{imp\_split}
which is an arbitrary split function that has the Hoare Triple
specified in \Cref{sec:imp-split}.

\begin{figure}
\begin{lstlisting}[mathescape=true, language=Isabelle, label={lst:imp-isin-fun},
    caption={The imperative isin function}]
partial_function (heap) isin :: 'a btnode ref option $\Rightarrow$ 'a $\Rightarrow$  bool Heap
  where
    isin p x =
  (case p of
     None $\Rightarrow$ return False |
     (Some a) $\Rightarrow$ do {
       node $\leftarrow$ !a;
       i $\leftarrow$ imp_split (kvs node) x;
       tsl $\leftarrow$ pfa_length (kvs node);
       if i < tsl then do {
         s $\leftarrow$ pfa_get (kvs node) i;
         let (sub,sep) = s in
         if x = sep then
           return True
         else
           isin sub x
       } else
           isin (last node) x
    }
)
\end{lstlisting}
\end{figure}

The Hoare Triple to argue that this is a refinement of the abstract operation
follows.
The convention for variables that refer to refined datatype
is that they have the same name as the abstract datatype,
with the postfix \textit{i}.
For example below, the abstract \textit{btree} is referred to by $t$,
however the refined \textit{btnode ref option} that represents it
is called $ti$.
To simplify notation, we simply assume that the set operations
such as \textit{isin} are overloaded and refer to abstract or
imperative version, depending on the parameter type.

\begin{lemma} $\sorted (\inorder t) \Longrightarrow$ \\
\begin{align*}
   \langle &\btreeassn k\ t\ ti\rangle \\
           &\isinfun ti\ x \\
   \langle \lambda r. &\btreeassn k\ t\ ti * \uparrow(\isinfun t\ x = r)\rangle_t
\end{align*}
\end{lemma}

Note how, in contrast to the abstract specification,
we need to also show that the imperative function does not modify the tree
residing in the heap.
The proof follows by induction on the computation of the abstract isin function.

% TODO conclusion ?
Inside a structured proof skeleton of the induction,
proving the subgoals in apply style turned out to be the most
appropriate.
The only help that the separation automation tool required
is a manually specification of the instantiation for existential quantifiers
for applying the inductive hypothesis. 
% TODO more details?

\subsection{Insertion}
% TODO...
\begin{figure}
\begin{lstlisting}[mathescape=true, language=Isabelle, label={lst:imp-nodei-fun},
    caption={The imperative node$_i$ function}]
definition node$_i$ 
    :: nat $\Rightarrow$ ('a btnode ref option $\times$ 'a) pfarray $\Rightarrow$ 'a btnode ref option $\Rightarrow$ 'a btupi Heap
  where node$_i$ k a ti $\equiv$ do {
      n $\leftarrow$ pfa_length a;
      if n $\le$ 2*k then do {
        a' $\leftarrow$ pfa_shrink_cap (2*k) a;
        l $\leftarrow$ ref (Btnode a' ti);
        return (T$_i$ (Some l))
      }
      else do {
        b $\leftarrow$ (pfa_empty (2*k)
            :: ('a btnode ref option $\times$ 'a) pfarray Heap);
        i $\leftarrow$ split_half a;
        m $\leftarrow$ pfa_get a i;
        b' $\leftarrow$ pfa_drop a (i+1) b;
        a' $\leftarrow$ pfa_shrink i a; 
        a'' $\leftarrow$ pfa_shrink_cap (2*k) a';
        let (sub,sep) = m in do {
          l $\leftarrow$ ref (Btnode a'' sub);
          r $\leftarrow$ ref (Btnode b' ti);
          return (Up$_i$ (Some l) sep (Some r))
        }
      }
    }

\end{lstlisting}
\end{figure}

\begin{figure}
\begin{lstlisting}[mathescape=true, language=Isabelle, label={lst:imp-ins-fun},
    caption={Excerpt of the imperative insert function}]

partial_function (heap) ins :: nat $\Rightarrow$ 'a $\Rightarrow$ 'a btnode ref option $\Rightarrow$ 'a btupi Heap 
    where 
      ins k x apo = (case apo of 
    None $\Rightarrow$ return (Up$_i$ None x None) | 
    (Some ap) $\Rightarrow$ do { a $\leftarrow$ !ap; i $\leftarrow$ imp_split (kvs a) x; tsl $\leftarrow$ pfa_length (kvs a); 
      if i < tsl then do { 
        s $\leftarrow$ pfa_get (kvs a) i; 
        let (sub,sep) = s in 
        if sep = x then return (T$_i$ apo) 
        else do { 
          r $\leftarrow$ ins k x sub; 
          case r of  
            (T$_i$ lp) $\Rightarrow$ do { pfa_set (kvs a) i (lp,sep); return (T$_i$ apo) 
            } | 
            (Up$_i$ lp x' rp) $\Rightarrow$ do { 
              pfa_set (kvs a) i (rp,sep); 
              if tsl < 2*k then do { 
                  ts' $\leftarrow$ pfa_insert (kvs a) i (lp,x'); 
                  ap := (Btnode ts' (last a)); 
                  return (T$_i$ apo) 
              } else do { 
                ts' $\leftarrow$ pfa_insert_grow (kvs a) i (lp,x'); 
                node$_i$ k ts' (last a) 
        } } } } else $\dots$

definition insert :: nat $\Rightarrow$ ('a::{heap,default,linorder}) $\Rightarrow$ 'a btnode ref option $\Rightarrow$ 'a btnode ref option Heap
 where insert $\equiv$ $\lambda$ k x ti. do {
  ti' $\leftarrow$ ins k x ti;
  case ti' of
     T$_i$ sub $\Rightarrow$ return sub |
     Up$_i$ l a r $\Rightarrow$ do {
        ts' $\leftarrow$ pfa_init (2*k) (l,a) 1;
        t' $\leftarrow$ ref (Btnode ts' r);
        return (Some t')
      }
}

\end{lstlisting}
\end{figure}

%TODO Description of applying the refinement framework to the insert function
The imperative implementation and derived heap rule of the insertion function follow
the same pattern as the isin function.
The only interesting part is whether and how often new memory is allocated.
An excerpt of the ins function can be seen in \Cref{lst:imp-ins-fun}.
The function includes some optimizations, such as not calling
\textit{node$_i$} if the node is not overflowing.
This is an optimization as otherwise \textit{node$_i$} in
\Cref{lst:imp-nodei-fun} would always allocate a new B-Tree node
rather than updating the current node in place.

To prove that this optimization is valid, the only obligation
is to show that the updated node itself is equivalent
to the node returned by \textit{node$_i$}.

\begin{lemma}$\length ts \leq 2k \Longrightarrow
    \node_i k\ ts\ t = T_i (Node\ ts\ t)$
\end{lemma}

Since we are in the branch where the node contains less than $2k$
elements, this is easy to show, provided some intermediate lemmas
that relate concrete node and abstract node in this context.
Adding a lemma on the equivalence between abstract and imperative
\textit{node$_i$}, we finally obtain a lemma for \textit{ins}
by induction on the computation
of the abstract \textit{ins} function.

\begin{lemma} $\sorted (\inorder t) \Longrightarrow$ \\
    \begin{align*}
       \langle &\btreeassn k\ t\ ti\rangle \\
               &\insfun k ti\ x \\
       \langle \lambda r. &\btreeassn k\ (\insfun k\ t\ x) r\rangle_t
    \end{align*}
\end{lemma}

The insertion function incorporates the function $\tree_i$
in much the same way.
Here the proof of equivalence followed automatically.
With the hoare rule for $ins$, we obtain the final desired lemma
on the imperative insertion.

\begin{theorem} $k > 0 \wedge \sorted (\inorder t) \Longrightarrow$ \\
\begin{align*}
   \langle &\btreeassn k\ t\ ti\rangle \\
           &\insertfun k\ x\ ti \\
   \langle \lambda ri. &\btreeassn k\ (\insertfun k\ x\ t)\ ri\rangle_t
\end{align*}
\label{thm:imp-ins-rule}
\end{theorem}

% TODO say something about k?
Note how we have the additional assumption that $k$ be positive.
This was only required to show correct order in the abstract setting.
Here, the capacity of the root node needs to be specified at initialization.
And since the capacity has to be $2k$ and the node has to hold at least one element

Until now we have only shown that the concrete
implementations refine the abstract interpretations.
What does this bring concretely?
Do we have to finally resort to an informal proof to show that the result
satisfies the desired properties?
This is not the case.
The theorem that would likely be desired by a purist\footnote{
    Admittedly, a purist that accepts our alternative Set interface.
    However this interface was shown to implement the standard Set interface
    so this should not really be an issue.
} would be the following

\begin{corollary} $k > 0 \Longrightarrow$
\begin{align*}
   \langle &\btreeassn k\ t\ ti \wedge \btree_k t \rangle \\
           &\insertfun k\ x\ ti \\
   \langle \lambda ri. \exists_A r. &\btreeassn k\ r\ ri \wedge \btree_k r \wedge \inorder r = \inslist x\ (\inorder t)\rangle_t
\end{align*}
\end{corollary}

And indeed, this can be derived automatically with a single command invocation,
using \Cref{thm:imp-ins-rule}, \Cref{thm:ins-set} and \Cref{thm:ins-invar}.\footnote{
    Technically we also need the fact that ins$_{list}$ preserves sortedness.
    This was already proven by Nipkow \parencite{DBLP:conf/itp/Nipkow16}
    and is considered trivial here.
}


\subsection{Deletion}

An imperative version of the deletion operation has been specified.
It required some additional operations on partially filled arrays but
posed no foundational challenges.
Due to its length it is not shown here but may be looked up in the
actual proof.

However the proof of its refining property was not conducted.
We assume that this proof would be structured similar to
the previous proofs.
The only benefit would be exploring usage of the new
heap rules introduced for the additional array operations.
The main difference to the verification of the
insertion function is that the refined functions
are undefined for many cases and hence it is highly important
to strongly tie the refined and imperative version
in order to discharge impossible cases.