% !TeX root = ../main.tex
\chapter{\abstractname}

The B-Tree data structure is a classical data structure
that lays the foundation of many modern database systems.
Since its first definition by Bayer \parencite{DBLP:journals/acta/BayerM72}
variations have been developed.
The original definition and modern variants are examined
as well as common variations of the deletion operation,
which varies strongly among implementations.
We derive our own definitions based on the original specification
and specifications that had been verified previously,
whether mechanically or with a pen and paper proof.
According to this definition, a functional version of the B-Tree
is implemented in the Isabelle/HOL framework,
an interactive automated theorem prover.
The implementation is complemented by a proof of its correctness
with respect to refining a set on linearly ordered elements
and a proof of the logarithmic relationship between height and number of nodes.
This reproduces results of Bayer \parencite{DBLP:journals/acta/BayerM72}
supporting claims on the efficiency of
operations on the tree.

From the functional specification an imperative version is
derived in Isabelle/HOL using refinement.
The main difference to the functional specification is the efficient
usage of a heap.
The implementation is kept general to support different
searching algorithms within nodes.
We provide a final version that works with a linear search
and one that supports binary search within the node.
These imperative programs are shown to again refine the functional specifications
using the separation logic utilities from the Isabelle Refinement Framework by
Lammich \parencite{DBLP:journals/jar/Lammich19}.
For this purpose, some additional imperative data structures
and operations on variants of arrays are introduced.
The result is a proof of the functional correctness
of an imperative implementation of the B-Tree data structure
that supports set membership and insertion queries
and uses efficient binary search for intra-node navigation.
The whole proof can be found in the appendix \parencite{MuendlerAppendix21}.

As with every specification in Imperative/HOL,
automatic executable code extraction to
several functional programming languages is supported.
This approach compares well to other approaches at mechanized
verifications of B-Tree implementations
considering development time and effort in lines of code.


%TODO: Abstract


