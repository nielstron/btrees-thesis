\PassOptionsToPackage{table,svgnames,dvipsnames}{xcolor}

\usepackage[utf8]{inputenc}
\usepackage[T1]{fontenc}
\usepackage[sc]{mathpazo}
\usepackage[ngerman,american]{babel}
\usepackage[autostyle]{csquotes}
\usepackage[%
  backend=biber,
  url=false,
  style=alphabetic,
  maxnames=4,
  minnames=3,
  maxbibnames=99,
  giveninits,
  uniquename=init]{biblatex} % TODO: adapt citation style
\usepackage{graphicx}
\usepackage{scrhack} % necessary for listings package
\usepackage{listings}
\usepackage{lstautogobble}
\usepackage{tikz}
\usepackage{pgfplots}
\usepackage{pgfplotstable}
\usepackage{booktabs}
\usepackage[final]{microtype}
\usepackage{caption}
\usepackage[hidelinks]{hyperref} % hidelinks removes colored boxes around references and links

\bibliography{bibliography}

\setkomafont{disposition}{\normalfont\bfseries} % use serif font for headings
\linespread{1.05} % adjust line spread for mathpazo font

% Add table of contents to PDF bookmarks
\BeforeTOCHead[toc]{{\cleardoublepage\pdfbookmark[0]{\contentsname}{toc}}}

% Define TUM corporate design colors
% Taken from http://portal.mytum.de/corporatedesign/index_print/vorlagen/index_farben
\definecolor{TUMBlue}{HTML}{0065BD}
\definecolor{TUMSecondaryBlue}{HTML}{005293}
\definecolor{TUMSecondaryBlue2}{HTML}{003359}
\definecolor{TUMBlack}{HTML}{000000}
\definecolor{TUMWhite}{HTML}{FFFFFF}
\definecolor{TUMDarkGray}{HTML}{333333}
\definecolor{TUMGray}{HTML}{808080}
\definecolor{TUMLightGray}{HTML}{CCCCC6}
\definecolor{TUMAccentGray}{HTML}{DAD7CB}
\definecolor{TUMAccentOrange}{HTML}{E37222}
\definecolor{TUMAccentGreen}{HTML}{A2AD00}
\definecolor{TUMAccentLightBlue}{HTML}{98C6EA}
\definecolor{TUMAccentBlue}{HTML}{64A0C8}

% Settings for pgfplots
\pgfplotsset{compat=newest}
\pgfplotsset{
  % For available color names, see http://www.latextemplates.com/svgnames-colors
  cycle list={TUMBlue\\TUMAccentOrange\\TUMAccentGreen\\TUMSecondaryBlue2\\TUMDarkGray\\},
}

% Settings for lstlistings
\lstset{%
  basicstyle=\ttfamily,
  columns=fullflexible,
  autogobble,
  keywordstyle=\bfseries\color{TUMBlue},
  stringstyle=\color{TUMAccentGreen}
}

\newtheorem{theorem}{Theorem}[section]
\newtheorem{corollary}{Corollary}[theorem]
\newtheorem{lemma}[theorem]{Lemma}

\usepackage{amsmath}
\DeclareMathOperator{\nodes}{nodes}
\DeclareMathOperator{\height}{height}
\DeclareMathOperator{\bal}{bal}
\DeclareMathOperator{\order}{order}
\DeclareMathOperator{\sorted}{split}
\DeclareMathOperator{\inorder}{inorder}

\DeclareMathOperator{\separators}{separators}
\DeclareMathOperator{\subtrees}{subtrees}

\DeclareMathOperator{\rootorder}{root\_order}
\DeclareMathOperator{\fullnode}{full\_node}
\DeclareMathOperator{\fulltree}{full\_tree}
\DeclareMathOperator{\slimnode}{slim\_node}
\DeclareMathOperator{\slimtree}{slim\_tree}
\DeclareMathOperator{\splitfun}{split}

\DeclareMathOperator{\emptyfun}{empty}
\DeclareMathOperator{\isinfun}{isin}
\DeclareMathOperator{\insertfun}{insert}
\DeclareMathOperator{\deletefun}{delete}
\DeclareMathOperator{\setfun}{set}
\let\div\relax
\DeclareMathOperator{\div}{div}

% TODO add all functions here
\lstset{emph={%
    linear_split,%
    linear_split_help,%
    isin,%
    split,%
    up,%
    ins,%
    insert,%
    tree,%
    node,%
},emphstyle={\upshape}
}
