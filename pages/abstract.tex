% !TeX root = ../main.tex
\chapter{\abstractname}

The B-tree data structure is a classical data structure
that builds the foundation of many modern database systems.
Since its first definition by Bayer \parencite{DBLP:journals/acta/BayerM72},
a plethora of variations of the data structure and operations on it have been developed.
We examine the original definition and modern alternative implementations,
with special focus on the deletion operation, which differs strongly among the literature.
We derive our own definitions, combining the original paper
with approaches that have resulted in verified implementations previously.
Based on the definition, we specify the B-tree data structure in the
functional modeling language HOL.
The specification is complemented by a proof of its correctness
with respect to refining a set of linearly ordered elements.
In addition, we provide a proof of the logarithmic relationship between height and number of nodes,
reproducing results of Bayer \parencite{DBLP:journals/acta/BayerM72}.
This verifies claims on the efficiency of
operations on the tree.
All proofs are machine-checked in the Isabelle/HOL framework,
an interactive automated theorem prover \parencite{DBLP:books/sp/NipkowK14}.
Within the framework,
the functional specification already yields automatic extraction of executable,
but inefficient code.

Therefore we derive an imperative implementation of the functional specification
in Imperative/HOL, using refinement.
The main difference is the efficient usage of a heap and arrays.
For this purpose, we introduce static arrays with variable size
and sophisticated copy and move operations on arrays.
The implementation is defined with respect to some abstract imperative
operation for node-internal search.
We provide one such function that implements a linear search,
and one that conducts binary search within the node.
These imperative programs are shown to again refine the functional specifications
using the separation logic utilities from the Isabelle Refinement Framework by
Lammich \parencite{DBLP:journals/jar/Lammich19}.
This process results in a proof of the functional correctness
of an imperative implementation of the B-tree data structure.
The implementation supports set membership and insertion queries
and uses efficient binary search for intra-node navigation.
As with every specification in Imperative/HOL,
automatic executable code extraction to
several functional programming languages is supported.

Our approach to verify an imperative implementation
via the detour of a functional specification
compares well to other approaches at mechanized
verifications of B-tree implementations
considering development time and effort in lines of code.
All proofs, specifications and programs can be found in the appendix \parencite{MuendlerAppendix21}.


%TODO: Abstract


