% !TeX root = ../main.tex
\chapter{\abstractname}

The B-Tree data structure is a classical data structure
that lays the foundation of many modern database systems.
Since its first definition in \parencite{DBLP:journals/acta/BayerM72}
variations of definition and basic operations have been developed,
the more complex operation deletion is spared by many
standard references.
The original definition and modern variants are examined and compared
with mechanically verified versions of the data structure.
A functional version of the B-Tree
was implemented in the Isabelle/HOL framework, an interactive automated theorem prover.
The implementation is complemented by a proof of its correctness
with respect to refining a set on linearly ordered elements
and a proof of the logarithmic relationship between height and number of nodes.
This reproduces results of \parencite{DBLP:journals/acta/BayerM72}
supporting claims on the efficiency of
operations on the tree.

From the functional specification an imperative version is
derived in the same framework.
The main difference to the functional specification is the efficient
usage of a heap.
The implementation is shown to again refine the functional program
using the separation logic utilities for the Isabelle Refinement Framework from
\parencite{DBLP:journals/jar/Lammich19}.
The result is a proof of the functional correctness
of an imperative implementation of the B-Tree data structure
that supports set membership and insertion queries.


%TODO: Abstract


