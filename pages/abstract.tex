% !TeX root = ../main.tex
\chapter{\abstractname}

In this thesis, we prove the functional correctness
of an imperative implementation of the classical B-tree data structure \parencite{DBLP:journals/acta/BayerM72},
supporting set membership and insertion queries
and using efficient binary search for intra-node navigation
in the interactive theorem prover Isabelle/HOL \parencite{DBLP:books/sp/NipkowPW02}.
This is done by first specifying the structure abstractly 
in the functional modeling language HOL and proving functional correctness.
Using manual refinement, we derive an imperative implementation
in Imperative/HOL. The correctness of this refinement is shown using
the separation logic utilities from the
Isabelle Refinement Framework by Lammich \parencite{DBLP:journals/jar/Lammich19}. 
The code can be exported to the programming languages SML and Scala.
The runtime is of all operations examined indirectly via a
proof of the logarithmic relationship between height and number of nodes.
Our approach to verify an imperative implementation
via the detour of a functional specification
compares well to other approaches at mechanized
verifications of B-tree implementations,
considering development time and effort in lines of code.


%TODO: Abstract


