% !TeX root = ../main.tex
% Add the above to each chapter to make compiling the PDF easier in some editors.


\chapter{Conclusions}\label{chapter:conclusion}

In this work, we obtained an imperative implementation
of B-Trees that is capable to handle queries whether a key is contained
and queries to insert new keys.
Concrete code has been obtained in the languages Scala and SML,
where the code generator may simply be extended to support other languages.
We have proven it functionally correct with the help of an abstract
specification that is refined by the imperative code.
The abstract specification fully implements
a Set interface in Isabelle, also including deletion of elements.


\section{Working in Isabelle}

One main source of frustration is finding the correct
functions and lemmas to apply for proofs.
While in natural proofs a certain kind of function and corresponding equalities can be
expected to be known by the reader.
However for machine checked proofs, as the only meaningful way,
all used lemmas need to be provided.
Now usually the Isabelle/HOL provides tools like sledgehammer
or \textbf{find\_theorems} in order to find useful lemmas for the concrete situation.
However each have their own drawbacks.

If the goal is only to simplify a given expression but not to
directly solve the result, sledgehammer is useless
as it only returns the applied lemmas for a successfully finalized proof.
This would have been especially useful for the apply style proofs of the
imperative implementation, that was mainly comprised of simplification steps.
Maybe the tools can be extended to return what lemmas lead to promising
simplifications of the the goal term to provide additional input to the user.

Secondly \textbf{find\_theorems} is useless if the current state
is simplified too much to not perfectly match the required lemma anymore.
As a concrete example, the $inorder$ of B-Trees was first defined
with "foldr map \textit{$\dots$ ts []}".
The proof system could not provide any useful lemmas on expressions of this form.
However there exists the standart function $concat$ that
simplifies to the above and has useful lemmas for catenation
and splitting.
It was only discovered close to the finalization of the thesis
when more standard terms for employed functions were searched for.
Since the available lemmas were key to a practical
implementation of the inorder set interface,
this meant a wasted workload of almost 2000 lines of additional proof.
The lesson learnt is to always look for alternative or equivalent
formulations of properties as well as lemmas,
to see what kind of existing lemmas may be employed.

% TODO note that the proof for  sortedness has been reduced from 2000 lines
% by the sorted (inorder ...) notation (and Nipkows auxiliary funs/lemmas)
% comparison standard approach and list approach
The proof of the standard set interface required a number of additional functions
comprising 36 lines of code.
The bigger part however is made up by the proofs required.
Including additional auxiliary lemmas, the whole proof of sortedness
and correct set operations required a staggering over 1900 lines of proof.
The proof was complex and hardly readable due to the 
extensive use of efficient solvers.
This is opposing 5 lines for the inorder and sorted function and
483 lines of proof required for the inorder approach,
which is mostly composed of easily understandable chains of equations.

% TODO note that node_i requires temporary allocation of an array that contains all elements.
% it would be more efficient to derive a specialized version that takes two arrays
% and redistributes between them evenly

% TODO this might be conclusion
The actual proofs in this work have been written to use only
minimal requirements on the input.
For example in \Cref{lem:splitmax-inorder}, rather than
requiring balancedness and $\order k$ for $k > 0$ on the full tree,
it is only required that the node has at least two children,
the last two have equal height
and that the last tree has the same, recursive property.
Using a more specific assumption was beneficial during the proof,
as the less restrictive requirements turn up in the
induction hypothesis again and are easier to satisfy that way.
However this also called for more lemmas that show that
the weaker property is satisfied in valid B-Trees.
Nevertheless, generating the most general lemmas usually turned out to be
very useful, for example in \Cref{lem:nodei-order}, yielding the reusability of $\node_i$ for
merging two nodes in deletion.
Another use case was the applicability of weaker properties
even when temporary violations
of the general invariant were observed during deletion,
such as in the \textit{del} function in \Cref{lst:del-def}.
%---------------------------------

\section{Future Work and Comparisons}

It should be noted at this point that the obtained code is not yet
truly optimized with respect to minimal memory allocation and access.
For that, some further refinements are necessary.
For example, the imperative \textit{node$_i$} function
requires temporarily allocating
an array that has enough space to contain
all elements that are supposed to be merged or split.
Optimizing it would maybe give a function that takes
as an argument two arrays and redistributes elements between them,
or an array and an element that is to be inserted.
This naturally increases the required amount of additional,
specialized heap rules and makes the proofs more complex.


% TODO comparison with other approaches
Both Malecha and Gidon report the proof for invariants
to be the most difficult properties to prove.
In out approach, we were able to reason about the abstract invariants
in the context of an abstracted specification,
significantly simplifying proof reasoning.
A difference can be observed in \label{fig:proof-comparison}
both with respect to the amount of 
time invested and the amount of manually provided proofs.

\begin{figure}
    \centering
    \begin{tabular}{l|c|c|c}
        \                & \parencite{DBLP:conf/popl/MalechaMSW10}$^{+}$ & \parencite{DBLP:journals/sosym/ErnstSR15}$^{+d}$ & Our approach \\
        \hline
        Functional code &   360      & \dots                 & 324  \\
        Imperative code &   510      & \dots                 & 170  \\
        Proofs          &  5190      & 350 + 1220\footnote{
            The proof is done integrating TVLA and KIV, and hence comprises
            interactive proofs and "interactions" with the automated prover.
            The first number is the number of added rules.
            The second number given here interprets each "interaction" as one apply-Style
            command and hence one line of proof.
        } & 2753 \\
        Timeframe (months) &  -     & 6+                      & 3   \\
    \end{tabular}
    \caption{Comparison of Lines of Code and time investment in related mechanized B-Tree verifications.
    $^+$ denotes implementations of B$^+$-Trees.
    $^d$ denotes that the implementation additionally verifies deletion operations.
    }
    \label{fig:proof-comparison}
\end{figure}

Finally it should be noted that this work implements B-Trees as
concretizations of sets.
However usually B-Trees are rather applied as concretizations
of maps, for example mapping the unique primary keys
of database tuples to the stored data.
Future work could adjust the code and proofs given here
to show that they can satisfy the map interface.
The main difference would then be that not the indices
are the main elements of the nodes but pairs of the 
index and the corresponding data.
